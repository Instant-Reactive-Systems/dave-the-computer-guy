\documentclass[12pt]{article}
\usepackage{amsmath}
\usepackage{amsfonts}
\usepackage{hyperref}
\usepackage{textcomp}
\usepackage{parskip}
\usepackage{graphicx}
\graphicspath{ {../images/} }
\hypersetup{
    colorlinks,
    citecolor=black,
    filecolor=black,
    linkcolor=black,
    urlcolor=black
}
\author{Andrija Milovac, Roko Burilo}
\title{Dave The Computer Guy}
\date{23.9.2021}
\begin{document}
\maketitle
\tableofcontents
\section{Product vision}
\subsection{Simple explanation}
Dave The Computer Guy is a WebGL game with the goal of teaching you how to build a computer from scratch out of logic gates step by step.
\subsection{Story intro}
You are in control of Dave, a fresh out-of-college computer science graduate.
After he graduated, he was hired by a local electronics manufacturer called "Electro Nick's".

He works in the company's office with his assigned mentor and senior hardware developer John Dodi and their boss Dexter Crawford.

Dave's day to day will consist of syncing with his boss about the work that needs to be done and doing the required work which is mostly
creating new electronic components and manufacturing them.

Whenever Dave gets stuck on a task or has no idea where to even start he can always ask John Dodi for help as he is a senior developer with
decades of experience and wisdom.

The reason he wants to do this is that he was originally a software engineer and only recently dived into the world of hardware and he
wishes to connect his software knowledge with his newly found hardware knowledge so that he can have a deep understanding of computers.

\subsection{Graphics}

The game will use webGL to display its graphics.\\
The graphics are 3D and voxel models will be used.\\
An isometric camera will be used to view the scenes.\\
There will be 3 main scenes in which the player will play:\\
\begin{enumerate}
    \item The office - Dave's workplace
          \begin{center}
              \includegraphics[width=5cm, height=4cm]{office.png}
          \end{center}
    \item Home - Dave's home where he spends time learning and creating.
          \begin{center}
              \includegraphics[width=5cm, height=4cm]{home.png}
          \end{center}
    \item Electronics shop - Dave buys circuit parts here from a salesman named Adem Shady
          \begin{center}
              \includegraphics[width=5cm, height=4cm]{home.png}
          \end{center}
\end{enumerate}

Each of the mentioned 3D scenes will contain clickable elements that will open additional UI elements where the gameplay will actually happen 
like for example the circuit editor window.

\subsection{Player models}

There are 3 NPC models and 1 player model.

The NPCs are:

\begin{enumerate}
    \item Dexter Crawford - the boss
    \item John Dodi - the senior engineer
    \item Adam Shady - the shop clerk
\end{enumerate}

The models can be viewed down below:

\begin{enumerate}
    \item Dexter Crawford - the boss
          \begin{center}
              \includegraphics[width=5cm, height=4cm]{boss.png}
          \end{center}
    \item John Dodi - the senior engineer
          \begin{center}
              \includegraphics[width=5cm, height=4cm]{john-dodi.png}
          \end{center}
    \item Adam Shady - the shop clerk
          \begin{center}
              \includegraphics[width=5cm, height=4cm]{clerk.png}
          \end{center}
    \item Player
          \begin{center}
              \includegraphics[width=5cm, height=4cm]{player.png}
          \end{center}
\end{enumerate}

\subsection{Gameplay}
The goal of the game is too build your own computer at home that can execute instructions of a predefined format (we made the instructions 
predefined so that we can validate the validity of the made computer by executing predefined programs).\\
The game will follow a linear path which is of the following form.\\
\begin{enumerate}
    \item Get an assignment from your boss.
    \item Solve the assignment. (this includes producing various components and buying the necessary parts to make the components)
    \item Get paid for solving the assignment and unlock new components.
    \item Use the knowledge you've acquired to advance your computer during your free-time at home.
    \item Once you progressed and unlocked every type of component you will get the final mission of building a computer.
    \item Once you build a computer that satisfies the given conditions (checked via tests made by the developers) you have officially completed the main story of the game.
\end{enumerate}

\section{Product market}
The audience for this product are electronics and computer enthusiasts as well as college students.\\
The product could be used by universities to teach computer architecture in a fun way engaging and rewarding way.\\
Profit will be earned by leveraging the game economy and reward ads. (Up to disscussion, detailed description needed).\\
\section{Product implementation}
The core simulator will be written in Rust (compiled into WASM) or in TypeScript (ran in a seperate worker).

The frontend of this application will be made using the following technologies.\\
\begin{itemize}
    \item Svelte (Sveltekit) as the frontend framework
    \item Threlte (Three.js) as the webGL framework (library)
    \item TailwindCSS for styling
    \item Google adsense ads
\end{itemize}
The backend of this application will be made using the following technologies.\\
\begin{itemize}
    \item Vert.x as the backend web framework (toolkit more precisely)
    \item MongoDB as the database
\end{itemize}
The application will be hosted on a VPS (most likely Oracle on an Oracle free VPS)
\end{document}
